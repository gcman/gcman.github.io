\PassOptionsToPackage{unicode=true}{hyperref} % options for packages loaded elsewhere
\PassOptionsToPackage{hyphens}{url}
%
\documentclass[10pt,letterpaper,]{article}
\usepackage{lmodern}
\usepackage{amssymb,amsmath}
\usepackage{ifxetex,ifluatex}
\usepackage{fixltx2e} % provides \textsubscript
\ifnum 0\ifxetex 1\fi\ifluatex 1\fi=0 % if pdftex
  \usepackage[T1]{fontenc}
  \usepackage[utf8]{inputenc}
  \usepackage{textcomp} % provides euro and other symbols
\else % if luatex or xelatex
  \usepackage{unicode-math}
  \defaultfontfeatures{Ligatures=TeX,Scale=MatchLowercase}
\fi
% use upquote if available, for straight quotes in verbatim environments
\IfFileExists{upquote.sty}{\usepackage{upquote}}{}
% use microtype if available
\IfFileExists{microtype.sty}{%
\usepackage[]{microtype}
\UseMicrotypeSet[protrusion]{basicmath} % disable protrusion for tt fonts
}{}
\IfFileExists{parskip.sty}{%
\usepackage{parskip}
}{% else
\setlength{\parindent}{0pt}
\setlength{\parskip}{6pt plus 2pt minus 1pt}
}
\usepackage{hyperref}
\hypersetup{
            pdftitle={Comparing e\^{}\textbackslash{}pi and \textbackslash{}pi\^{}e},
            pdfauthor={Gautam Manohar},
            pdfborder={0 0 0},
            breaklinks=true}
\urlstyle{same}  % don't use monospace font for urls
\setlength{\emergencystretch}{3em}  % prevent overfull lines
\providecommand{\tightlist}{%
  \setlength{\itemsep}{0pt}\setlength{\parskip}{0pt}}
\setcounter{secnumdepth}{5}
% Redefines (sub)paragraphs to behave more like sections
\ifx\paragraph\undefined\else
\let\oldparagraph\paragraph
\renewcommand{\paragraph}[1]{\oldparagraph{#1}\mbox{}}
\fi
\ifx\subparagraph\undefined\else
\let\oldsubparagraph\subparagraph
\renewcommand{\subparagraph}[1]{\oldsubparagraph{#1}\mbox{}}
\fi

% set default figure placement to htbp
\makeatletter
\def\fps@figure{htbp}
\makeatother

\providecommand{\subtitle}[1]{%
  \usepackage{titling}
  \posttitle{%
    \par\large#1\end{center}}
}
\let\Pr\relax
\DeclareMathOperator\LCM{LCM}
\DeclareMathOperator\Pr{P}
\DeclareMathOperator\arccot{arccot}
\newcommand{\Mod}[1]{\ (\mathrm{mod}\ #1)}
\renewcommand{\vec}[1]{\mathbf{#1}}
\newcommand{\norm}[1]{\left\lVert#1\right\rVert}

\newcommand{\inn}[1]{\in\left\{#1\right\}}
\newcommand{\ninn}[1]{\not\in\left\{#1\right\}}
\newcommand{\orr}{\qquad&\,\text{or}\qquad}

\renewcommand\d{\mathop{}\!\mathrm{d}}
\newcommand{\diff}[2]{\frac{\d #1}{\d #2}}

\title{Comparing \(e^\pi\) and \(\pi^e\)}
\author{Gautam Manohar}
\date{2 February 2018}

\begin{document}
\maketitle
\emph{This document originally appeared as a blog post on my website. Find it at} \texttt{\href{http://gautammanohar.com/e-pi.html}{gautammanohar.com/e-pi.html}}

The constants \(e\) and \(\pi\) are everywhere in mathematics.
Determining the greater of the two expressions \(e^\pi\) and \(\pi^e\)
(without using a calculator, of course\ldots{}) is a fun puzzle that you
can approach in many ways. I'd like to discuss my solutions.

\hypertarget{differentiation}{%
\section{Differentiation}\label{differentiation}}

We shall perform the same operations on the two expressions:
\begin{equation}
    \begin{split}
        e^\pi &\odot \pi^e \\
        e^{\frac{\pi}{e}} &\odot \pi^{\frac{e}{e}} \\
        e^{\frac{1}{e}} &\odot \pi^{\frac{1}{\pi}}.
    \end{split}
\end{equation} To show that \(e^\pi > \pi^e\), it suffices to show that
\(e^{\frac{1}{e}} > \pi^{\frac{1}{\pi}}\). Let \(y = x^{\frac{1}{x}}\).
Then we can implicitly differentiate to find the critical points.
\begin{equation}
    \begin{split}
        \ln{y} &= \ln{x^{\frac{1}{x}}} \\
        \ln{y} &= \frac{\ln{x}}{x} \\
        \diff{}{x}\ln{y} &= \diff{}{x}\frac{\ln{x}}{x} \\
        \frac{1}{y}y' &= \frac{x\left(\frac{1}{x}\right) - \ln{x}\cdot1}{x^2} \\
        y' &= x^{\frac{1}{x}}\frac{1 - \ln{x}}{x^2}.
    \end{split}
\end{equation} The expressions \(x^{\frac{1}{x}}\) and \(x^2\) are
always positive, so there is only critical point: when
\(1 - \ln{x} = 0\), or when \(x = e\). We must find whether this point
is a global minimum or a maximum. When \(x = 1 < e\), we have
\(1 - \ln{x} = 1\), so the function is increasing. The value
\(x = e^2 > e\) gives \(1 - \ln{x} = -1\), which means the function is
decreasing. Thus \(x^{\frac{1}{x}}\) has a global maximum at \(x = e\).
And so \(e^{\frac{1}{e}} > \pi^{\frac{1}{\pi}}\), which shows that
\(e^\pi > \pi^e\).

\hypertarget{inequality}{%
\section{Inequality}\label{inequality}}

If we use the inequality \(1 + x < e^x\) (I will present three proofs of
this below), then a very simple solution presents itself. The equality
holds for all \(x\), but we only require it to hold for positive \(x\).
Make the substitution \(x = \frac{\pi}{e} - 1\), in an effort to cancel
out the 1 on the right side of the inequality and introduce \(\pi\).
Because \(\pi>e\), \(\frac{\pi}{e} - 1 > 0\), and so \begin{equation}
    \begin{split}
        1 + \frac{\pi}{e} - 1 &< e^{\frac{\pi}{e} - 1} \\
        \pi\cdot\frac{1}{e} &< e^{\frac{\pi}{e}} \cdot \frac{1}{e} \\
        \pi &< e^{\frac{\pi}{e}}\\
        \pi^e &< e^\pi.
    \end{split}
\end{equation} Wonderful, isn't it?

\hypertarget{taylor-series}{%
\subsection{Taylor Series}\label{taylor-series}}

This is the most standard proof I have; I think it's the least exciting.
We only prove the equality for positive \(x\). We know \begin{equation}
    e^x = 1 + x + \frac{x^2}{2!} + \frac{x^3}{3!} + \cdots
\end{equation} Thus for \(x > 0\), all the terms on the right side will
be positive, and so \(e^x > 1 + x\).

\hypertarget{concavity}{%
\subsection{Concavity}\label{concavity}}

This method requires a little more ``geometric intuition'' than the
last. At \(x = 0\), we have \(e^x = 1\). At this point \((0,1)\), the
tangent line to \(e^x\) has slope \(1\) and has the equation
\(y = 1 + x\). Because \((e^x)'' = e^x > 0\), \(e^x\) is always concave
up, so it is always above its tangent line. Therefore, \(1 + x < e^x\).

\hypertarget{am-gm}{%
\subsection{AM-GM}\label{am-gm}}

This is my favourite proof. It's a little less intuitive than the
others, but I think it's beautiful. We use the arithmetic-geometric mean
inequality. \begin{equation}
    \begin{split}
        \sqrt[n]{1+x} &= \sqrt[n]{\smash[b]{\underbrace{1\cdot1\cdots1}_{\text{$n-1$ times}}\cdot(1+x)}} \\[1em]
        &\le \frac{\overbrace{1+\dotsb+1}^{\text{$n-1$ times}}}{n} \\
        &= \frac{\overbrace{1+\dotsb+1}^{\text{$n-1$ times}}+(1+x)}{n} \\
        &= \frac{\overbrace{1+\dotsb+1}^{\text{$n$ times}}+x}{n} \\
        &= 1 + \frac{x}{n}.
    \end{split}
\end{equation} Strict equality in the AM-GM inequality only holds when
all the terms are equal. In this case, \(x > 0\), so \(1 + x \neq 1\),
so we have strict inequality. This gives
\(\sqrt[n]{1+x} < 1 + \frac{x}{n}\). Raising both sides to the \(n\)-th
power gives \begin{equation}
    1 + x < \left(1 + \frac{x}{n}\right)^n.
\end{equation} Taking the limit as \(n\) approaches \(\infty\) on both
sides yields \begin{equation}
    \begin{split}
        \lim_{n\to\infty} (1 + x) &< \lim_{n\to\infty} \left(1 + \frac{x}{n}\right)^n \\
        1 + x &< e^x.
    \end{split}
\end{equation} We use the limit definition of \(e\) to conclude our
proof.

I get the feeling that there are many other ways to attack this problem.
If you can solve it with a method that I have not shown, please let me
know!

\end{document}