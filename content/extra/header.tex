\PassOptionsToPackage{colorlinks=true, urlcolor=MidnightBlue}{hyperref}
\providecommand{\subtitle}[1]{%
	\usepackage{titling}
	\posttitle{%
	\par\large#1\end{center}}
}
\lstset{
	aboveskip={1.3\baselineskip},
	basicstyle=\ttfamily\linespread{4},
	breaklines=true,
	columns=flexible,
	commentstyle=\color[rgb]{0.127,0.427,0.514}\ttfamily\itshape,
	escapechar=@,
	escapeinside={\%*}{*)},
	extendedchars=true,
	identifierstyle=\color{black},
	inputencoding=latin1,
	keywordstyle=\color[HTML]{228B22}\bfseries,
	language=Python,
	ndkeywordstyle=\color[HTML]{228B22}\bfseries,
	prebreak = \raisebox{0ex}[0ex][0ex]{\ensuremath{\hookleftarrow}},
	showstringspaces=false,
	stringstyle=\color[rgb]{0.639,0.082,0.082}\ttfamily,
	upquote=true,
	otherkeywords={self},
	emph={MyClass,__init__},
}

\usepackage{mathrsfs}
\usepackage{mhchem}
\usepackage[labelformat=empty]{caption}

\let\Pr\relax
\DeclareMathOperator\LCM{LCM}
\DeclareMathOperator\Pr{P}
\DeclareMathOperator\arccot{arccot}

\DeclareMathOperator{\I}{\mathcal{I}}

\newcommand{\Mod}[1]{\ (\mathrm{mod}\ #1)}
\renewcommand{\vec}[1]{\mathbf{#1}}
\newcommand{\norm}[1]{\left\lVert#1\right\rVert}
\newcommand{\abs}[1]{\lvert #1 \rvert}

\newcommand{\inn}[1]{\in\left\{#1\right\}}
\newcommand{\ninn}[1]{\not\in\left\{#1\right\}}
\newcommand{\orr}{\qquad&\,\text{or}\qquad}

\renewcommand\d{\mathop{}\!\mathrm{d}}
\newcommand{\diff}[2]{\frac{\d #1}{\d #2}}
\newcommand*{\qed}{\hfill\ensuremath{\square}}

\renewcommand\mathscr{\mathcal}